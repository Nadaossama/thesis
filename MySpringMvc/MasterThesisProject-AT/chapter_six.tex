\chapter{Evaluation}
\label{ChapterSix}




%\indent The main area of focus of the current thesis was offline database synchronization algorithms for mobile applications. The work was split in three main steps. The first part included researching and documenting the state-of-the-art in this area. As many research directions were found, only the main approaches were presented.\\
%\indent Offline database synchronization was in the attention of the developers and researchers especially after the development of distributed software systems. Mobile device applications are just an example of area in which this concept is used. However, for this research area, the challenges are greater, as mobile devices are still constrained to some limitations. On the other hands, these small gadgets are providing feature and functions that cannot be found on desktop applications. The use of GPS localization system or of the integrated camera for easily capturing images are just some examples.\\
%\indent The application built for Android OS contains the time stamp synchronization algorithm for SQLite mobile databases adjusted to the needs of the application. Therefore, an initial database capture synchronization is performed and it is then followed by periodic synchronizations in which inconsistent data conflicts are solved, and databases are update accordingly.\\
%\indent Database synchronization process was the most important part of the mobile application, but besides that, other subtle details (dependency on Internet connection, use of camera) had to be handled.\\
%\indent All the research work led to a proposal of a synchronization framework for distributed databases. The final practical result of the thesis work is an Android mobile application addressed for a B2B problem whose main focus was on the ordering process. 


%\indent Working for an application that will be deployed in the business environment is an important experience in learning the business mechanisms behind the development of a software system. Besides the actual programming of the application, other type of work has been conducted. It included the analysis of the business domain, understanding the requirements of the stakeholders and building the specifications (design, architecture) of the application according to both functional and non-functional requirements.\\
%\indent The development process could be started only after having defined the business domain. The application has been build incrementally, adjusting the development to the requirements and gathering continuous feedback from the representatives of the company.\\
%\indent The value of the application was given by the research work conducted in the area of the database synchronization. Studying the existent work is the most important point in defining a solution for the given problem. State-of-the-art gives a starting point and an overview on which algorithms exist, what are their advantages and drawbacks. By knowing this, one can propose a solution that minimizes the drawback and fits in to the domain constraints.\\
%\indent The integration of the database synchronization solution in the mobile application was slightly different from the expectation as no configuration could be done on the server side, and the client side had to implement most of the processing methods.\\
%\indent As we could notice, starting with an overview of both the theoretical part and of the domain definition, gives an image on how the system should be implemented, what could be the possible problems that can be encountered. These aspects help in avoiding big design flaws that can appear later in the application and that are more costly than a intense initial analysis.



%\indent The solution and the research work started in this master thesis can be further on continued in several directions. Firstly, as our main focus on the practical part was on the client side, aspects related to the server side might not have been extensively investigated and documented. Additional work in performing the necessary checks for discovering the inconsistent data on the server side can be carried on. The advantages of shifting the data records verification on the server side are mostly related to reducing the computational efforts.\\
%\indent Future directions of research can be headed towards new algorithms like Symmetric DS. It seems promising as it can be ported on applications running on different OSs and it handles automatically the synchronization of database information in case no other specifications are given. The current drawback is that it requires manual configuration of every node that is added in the network. This configuration has to be performed on the server side. Automatizing this process, will make it a powerful algorithm for cross-platform applications.\\
%\indent The implementation of the solution presented in chapter four of the thesis described the mobile application built for Android devices. The client side of the application can be built for other mobile OS like Windows Phone or iOS. The guidelines for developing the application for other OS are given in the chapter three in the presentation of the general solution. The application should follow the same idea on all the platforms, only the look and feel might slightly change in order to fulfill the expectations of the users.\\
%\indent The synchronization algorithm might be improved by searching for differences only in records' fields so few information is exchanged. Also, conflicts resolution is an important aspect that can be improved. By using time stamp algorithms, the last updated version of an entry is kept. In the future, better comparison methods can be developed so that the instances of a database can be kept in a more consistent state. Also, the framework might be configured in such way that conflict resolution can be configured according to the needs of the application.\\
%\indent Switching between offline and online working mode can lead sometimes to data loss. It is important to detect methods that can increase the offline working mode time and reduce the dependency on the Internet connection. For instance, reducing the amount of data that needs to be exchanged and proposing an algorithm to check the need of synchronization can be built. The synchronization rate can be learn incrementally from the update rate of the information in the database.\\
%\indent The research topics on the area of database synchronization will be always a subject that worth the attention as databases are essential parts of applications. Databases are essential in storing applications' information and the correctness and consistency of their data is the most important aspect that concerns them. For this reason, database synchronization subject remains an open area of research.