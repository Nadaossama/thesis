\chapter{Evaluation}
\markright{Evaluation}
\label{ChapterSix}
In this chapter we will discuss how our system could be evaluated. In \textit{ChapterTwo} we discussed the different methods used to evaluate decision support systems. Project Connect Hydro is still in the conceptual stage, consequently our web portal is making use of the research already done in the feasibility study to produce the first prototype. The prototype is developed as a proof of concept to prove that a decision support system comprised of two different types could benefit the connect hydro project and optimize the power plants energy production as well as minimize down times and maintenance fees.

Since our system didn't have any users, there was no possibility to perform user acceptance test or use their feedback for evaluation purposes. Our only feedback was received when our web portal was demoed to some power plant operators and they showed interest in such a system and were willing to try using it in the future as it becomes more comprehensive and in a production stage. Therefore, evaluation was reduced to checking that all the functionalities are included and are performing as expected. The functionalities include:
\begin{itemize}
	\item Create a DSS support system.
	\item Combine two different types of DSS.
	\item Integrating the event matrix into our system.
	\item Providing Data Visualization to the power plant owners.
	\item Allow for rule addition in an easy dynamic way for the knowledge-driven part of our system.
	\item Rules created by the operators were applied correctly as expected and the results were the expected ones.
\end{itemize}
In Connect Hydro, a qualitative benefit study was conducted through meetings with experts and power plants operators. In the study a few situations were studied and discussed that if a decision support system could prevent these events from happening then it will be a great interest to them and will motivate them to implement the system and help convince other operators to join the project. These situations include:
\begin{itemize}
	\item Too much Sand in the Canal: if too much sand accumulates, expensive special equipment need to be rented to remove and the power plant can experience a down time up to three months, thus losing too much money.
	\item High Turbidity : If the water is unclean, waste and leaves can get stuck in the turbines. Cleaning the turbines should not be a regular event but in the case of high turbidity it will need to become regular thus leading to more down time to maintain the turbines.
\end{itemize}
Our decision Support system should prevent these events from happening because of the early warning system. The system warns the power plant owners if the water coming into their power plant is unclean so that they can close the channel or turn off the turbines, the cost of turning it off until the water clears is much less than maintaining the turbines. In the case of too much sand, an event can be added in the event matrix that states that if a certain event leads to sand being in the water, the water channel should be closed. These are just some of the future benefits that can be introduced with our decision support system.

The Data visualization proved be an interesting tool that could help our system predict future trends in energy production based on the history as well as some external resources like weather forecasting