\chapter{System Architecture and Details}
\label{ChapterFive}
\section{Frameworks and Technologies}
\label{FrameworksAndTechnologies}
In this chapter, different techniques and frameworks used in web development will be introduced. Frame- works are divided into two types: Front-end and Back-end frameworks. The difference between them will be explained in addition to giving examples of the latest and newest frameworks for each of them.
\subsection{Web Development}
Web development is growing fast and the need for efficient and easy techniques for maintaining code and also improving user experience. There are a lot of different techniques used in web development such as Responsive Web Design, Ajax calls and web sockets.
Responsive Web Design [RWD] as shown in figure 2.1 is the concept of viewing the content of a website differently based on the screen size, platform and orientation of the device. The idea behind RWD depends on the usage of different grids and layouts. Also, loading of images based on the screen size of devices. Efficient use of CSS media queries. The website is developed in a way that it can automatically present the content of the website based on the resolution of the device and hence a user can use his/her laptop, tablet and even smart phone to open the website [8].
\subsection{Front-End Technologies}
Websites are becoming more and more important. They are growing rapidly, they are now used in marketing, sales, social media as well. This shows the need of developing rapid websites for different usages. unfortunately, building a website is a becoming a challenge due to the fact that nowadays, there are a lot of devices that can access the internet which have different screen sizes. Running a website from a smart phone is completely different from a desktop computer. This is one issue, another problem is supporting the different browsers. There are now Internet explorer, Mozilla Firefox, Chrome, Safari and many other browsers. Supporting them is not an easy task.
Front-end frameworks use HTML[Hypertext Markup Language], CSS and Javascript which are the main components for presenting content to user from a webpage. HTML is the content of the page itself, CSS is used to format and style the content of the page while Javascript is used for dynamic elements such as expanding menus and also to create events based on user interactions such as clicking on a button. Front-end framework is a platform where ready components can be used by the developer when working on a project, it also gives the develeoper the ability to customize such components. There are a lot of front-end frameworks such as Bootstrap, Foundation, AngularJS and many others. In this paper, only Bootstrap and AngularJS will be introduced because each one of them provides the developer with different features. More examples on front-end frameworks can be found on this webpage[12].
\subsubsection{Asynchronous Javascript And Xml}
Asynchronous Javascript And Xml known as ”AJAX”, it is the technique of updating the content of a webpage without the need to refresh the browser. This technique improves the user experience since it is more efficient. It improves the performance since it does not load the content already loaded at the user’s side but instead it updates the user’s page with the new content from the server. The idea is simply asynchronously exchanging small amounts of data with the server behind the scenes, without affecting the rest of the page. AJAX uses XML or JSON for getting data from server, JSON is used more often nowadays due to its native compatibility with Javascript. Finally, it uses CSS for styling and Javascript for prcocessing of requests from server and for updating page content as well [9].
\subsubsection{Bootstrap}
Bootstrap∗ was created by Mark Otto and Jacob Thornton, developers working at Twitter. The idea behind Bootstrap is to have a tool capable of helping in rapid development of web applications. Bootstrap helps developers to have consistant code that is easy to maintain. Bootstrap provides ready CSS classes and HTML components like form elements, tables and images. Bootstrap is also known for its layouts and grips where it helps in developing responsive web applications. Bootstrap uses JQuery library which a fast, compact Javascript library that helps in manipulating the HTML DOM in addition to event handling, Ajax calls with API to work with different browsers[13].
Using Bootstrap has a lot benefits. First, it is really easy to use even for a beginner. It is quick to learn since it provides an easily learning curve. It can be easily integrated all most with every framework. It is used for rapid development since it provides the developers with ready-made coding blocks which can be utilized to build a website easily. One of the popular features in Bootstrap is its grip system[14]. It provides developers with 12 column grid style as shown in figure 2.5 which helps a lot when developing a web responsive application[15]. One of the main reasons why Bootstrap is successful is because of its Detailed explanations and excellent documentation which help beginners as well as experienced developers in addition to its large coummunity. Like every tool, it has its drawbacks as well. Bootstrap’s main disadvantage is that websites using Bootstrap look the same.Twitter Bootstrap helps in rapid development of websites but it comes at the price of creativity which leads to the problem that all websites using Bootstrap components look alike. Some visitors feel that companies which use Bootstrap- built websites are not putting enough effort to have their custom design[16]
\subsection{Back-End Technologies}
Back-end consists of three parts: server, database and an application[22]. For a web application, it starts with the user opening a webpage where the content of the webpage is shown as an HTML with styling using CSS and some Javascript for dynamic content. The data provided by the website is static if there is no back-end. Back-end is the database containing all the data provided by the website in addition to the server which processes the user requests and respond with the right information. Back-end Frameworks are working with what clients do not see, Server and database mainly. They are responsible for logic and data manipulation. Back-end Framework is the one communicating with the database. It communicates with front-end by sending information to be displayed as a web page. when a user fills a form and submits, the back-end is the one taking responsibility of this request and respond to front-end again with new content to show to user based on this action or request[23]. Back-end frameworks facilitates and makes development faster. Three frameworks will be introduced: Java Spring, Ruby on Rails (RoR) and MeteorJS. Each one of them is chosen for certain reasons which will be discussed in details.
\subsubsection{Spring}
Spring Framework is one of the very popular open source frameworks that uses Java for building web and enterprise applications. It was initially written by Rod Johnson and was first released under the Apache 2.0 license in June 2003, that’s why it is popular and has a great and supportive community. Spring provides developers with a lot of features that will be discussed in details. One of the advantages is that it gives flexibility to its developers. For instance, developers can configure beans which are the objects that form the backbone of a Spring application. There are three ways for configuring beans: using XML, Annotations or JavaConfig. Spring framework was created primarily as a Dependency Injection container. Using dependency injection helps in testing the developer’s code. Spring uses MVC pattern as well and has support for both relational and NOSQL databases. Spring uses Java which is fast compared to frameworks using script-languages like Ruby on Rails. Spring uses robust frameworks for security and finally it integrates with a lot social networking sites like Facebook, Twitter, Github, etc. For beginners Spring can be complex due to its flexibility. The developer has multiple choices for the same task which makes it confusing. Spring is most identified by its dependency injection[38].

Dependency Injection is really essential when developing a complex Java application where a lot of classes are dependant on each other. What DI is trying to achieve is to eliminate depenencies between classes as much as possible. This makes it possible to reuse classes and to test them indepenently. For example, there are two classes A and B where Class A depends on Class B. DI will inject Class B into Class A by using Inversion of Control (IoC) as shown in figure 2.16. Dependency Injection can be applied in two ways: the class (B) is given to class (A) via the constructor of class A, this approach is called construction injection while via a setter is called setter injection[39].
Spring is a powerful framework but has its own disadvantages as well. Spring is a very complex framework which has more that 2400 classes, 49 other tools which complicate development. It needs a lot of configurations which makes it hard to learn. Spring will require developers to work and code with a lot of XML. Spring offers developers a lot of parallel mechanisms which confuses developers and needs a lot of understanding for choosing the best one for the current scenario the developer is implementing. All of these disadvantages makes development take a lot of time compared to other technologies like MeteorJS. However, Spring still offers a lot in return. Using Java makes it faster compared to a lot other frameworks. Spring achieves the loose coupling through dependency injection and interface based programming which prevents developers from writing messy code[40].
\subsubsection{MySQL Database}
\section{Database Design and Implementation}
\label{DatabaseDesignandImplementation}
\section{Web Portal}
\label{WebPrtal}
\section{Data Visualization}
\label{DataVisualization}
\section{Decision Support}
\label{DecisionSuport}