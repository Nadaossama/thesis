\chapter{Connect Hydro Project}
\markright{Connect Hydro Project}
\label{ChapterThree}
\section{Small Hydro Power Plants}
\label{SmallHydroPowerPlants}
Hydropower systems are systems that make use of the energy in flowing water to produce electricity or mechanical energy. The water flows via channel or penstock to a waterwheel or turbine where it strikes the bucket of the wheel, causing the shaft of the waterwheel or turbine to rotate. When generating electricity, the rotating shaft, which is connected to an alternator or generator, converts the motion of the shaft into electrical energy. This electrical energy may be used directly, stored in batteries, or inverted to produce utility-quality electricity. A small scale hydroelectric facility requires that a sizable flow of water and a proper height of fall of water, called head, is obtained without building elaborate and expensive facilities. Small hydroelectric plants can be developed at existing dams and have been constructed in connection with river and lake water-level control and irrigation schemes. By using existing structures, only minor new civil engineering works are required, which reduces the cost of this component of a development.

small scale hydropower systems capture the energy in flowing water and convert it to usable energy. Although the potential for small hydro-electric systems depends on the availablity of suitable water flow, where the resource exists it can provide cheap clean reliable electricity. A well designed small hydropower system can blend with its surroundings and have minimal negative environmental impacts.

Moreover, small hydropower has a huge, as yet untapped potential in most areas of the world and can make a significant contribution to future energy needs. It depends largely on already proven and developed technology, yet there is considerable scope for development and optimization of this technology.

Hydroelectricity is the term referring to electricity generated by hydropower; the production
of electrical power through the use of the gravitational force of falling or flowing water.
It is the most widely used form of renewable energy, accounting for 16 percent of global
electricity generation – 3,427 terawatt-hours of electricity production.
The main aspects of hydro power plants:
1) Design and Operation.
2) Generating methods.
3) Advantages and Disadvantages.

\subsection{Benefits of Small Hydro Power Plants}

Hydro electric energy is a continuously renewable electrical energy source. It is non-polluting - no heat or noxious gases are released. It is essentially inflation proof due to its low operating and maintenance cost, it also doesnt have any fuel cost. It is a proven technology that offers reliable and flexible operation along with a long life, many existing stations have been in operation for more than half a century and are still operating efficiently.
Hydro power station provide an efficiency of over 90\%, it the most efficient of energy conversion technologies. Finally, Hydro power offers a quick means of responding to changes in load demand or due to certain events.

\subsection{Generating methods}
\begin{itemize}
\item \textbf{Conventional (dams)} – Most hydroelectric power comes from the potential energy of dammed water driving a water turbine and generator. The power extracted from the water depends on the volume and on the difference in height between the source and the water's outflow. This height difference is called the head. The amount of potential energy in water is proportional to the head. A large pipe (the "penstock") delivers water to the turbine. 
\item \textbf{Pumped-storage} – This method produces electricity to supply high peak demands by moving water between reservoirs at different elevations. At times of low electrical demand, excess generation capacity is used to pump water into the higher reservoir. When there is higher demand, water is released back into the lower reservoir through a turbine. Pumpedstorage schemes currently provide the most commercially important means of large-scale grid energy storage and improve the daily capacity factor of the generation system. Pumped storage is not an energy source, and appears as a negative number in listings.
\item \textbf{Run-of-the-river} – Run-of-the-river hydroelectric stations are those with small or no reservoir capacity, so that the water coming from upstream must be used for generation at that moment, or must be allowed to bypass the dam.
\item \textbf{Tide} – A tidal power plant makes use of the daily rise and fall of ocean water due to tides; such sources are highly predictable, and if conditions permit construction of reservoirs, can also be dispatchable to generate power during high demand periods. Less common types of hydro schemes use water's kinetic energy or undammed sources such as undershot waterwheels. Tidal power is viable in a relatively small number of locations around the world.
\item \textbf{Underground} – An underground power station makes use of a large natural height difference between two waterways, such as a waterfall or mountain lake. An underground tunnel is constructed to take water from the high reservoir to the generating hall built in an underground cavern near the lowest point of the water tunnel and a horizontal tailrace taking water away to the lower outlet waterway.
\end{itemize}
% In this context, it is important to mention that small hydro power (SHP) plants are the most prosperous way for additional hydro power penetration in European electricity market, considering that most large-scale opportunities have either been already exploited or face serious contradictions by local societies as environmentally unacceptable (Kaldellis, 2002a, b). On the other hand, SHP units usually operate as “run-of-river” systems, thus any dam or barrage used is quite small, not really disturbing the water flow rate. Although to date there is no internationally agreed definition of SHP plant size, the official size in the local electricity generation market is set equal to 10 MW maximum (law 2244/94).
\subsection{Problems with Small Hydro Power Plants}
\section{Control Strategies}
\subsection{Local Control Strategy}
\subsection{Cooperative Control Strategy}
\subsection{Centralized Control Strategy}

\section{Connect Hydro Problem}
\subsection{Technical \& Financial Aspects}
\subsection{Proposed Concept}

\section{Proposed Implementation}
\subsection{Technical Level}
\subsection{Software Level}