\chapter{Introduction}
\markright{Introduction}
\label{ChapterOne}
\section{Reading Instructions}
\label{sec:Reading Instructions}
\indent This thesis may present different interests for different readers. In this chapter, I will provide a guideline explaining what is covered in each chapter in order to facilitate browsing of the thesis and efficiently help every reader find the relevant information for him/her. 

\indent The first chapter presents the details and circumstances in which this thesis was created upon. The Problem statement of this thesis will be defined along with the motivation for solving that specific problem. Readers interested in a high level overview of the goal and the approach used for solving the specified problem, along with the original contribution brought to the existing system should refer to the \textit{GoalAndApproach} and \textit{OriginalContribution} sub chapters respectively.

\indent The second chapter will discuss the most relevant concept of this thesis, namely decision support systems. They will be discussed and evaluated in terms of the foundations they are built on, their functionality, the Interfaces used for them, how they are implemented and their evaluation matrices and impact on decisions. This chapter will be most relevant for readers who would like to learn about decision support systems and understand the underlying concepts.

\indent The third chapter will cover the Connect Hydro Project that my thesis aims to support and add to it. Connect Hydro proposes a system to connect small, private and independent hydro power plants through networked intelligent control system. In the chapter, I will also give an overview on the device they developed to collect sensor data from the power plants.

\indent Chapter four will highlight in detail how a decision support system can bring advantage to the connect hydro project. In this chapter, I will also discuss what are the requirements for this proposed decision support system and describe the different inputs along with the expected outputs in addition to what should be the defined rules for such system. This is the chapter that my work will be based on.

\indent The fifth chapter will cover the technical aspects of the implementation done to support this thesis. It will begin with describing the frameworks and technologies used for the implementation while explaining why they were used. Furthermore, each implemented aspect of the project will be explained in detail, namely the database model, the web portal, the data visualization and finally and most importantly the decision support system. This chapter might be of interest also for readers that want to find more details about the design and implementation of this system.

\indent Chapter six will explain how the system implemented was evaluated, what matrices were used in its evaluation and the results. Readers interested in the results only will find this chapter the most informative for them.

\indent The last chapter containing the conclusion and the future research will be most relevant for users interested in extending and improving the proposed system.
% \section{Foreword}
% \label{sec:Foreword}
\section{Motivation}          
\label{sec:Motivation}
\indent Renewable energy is the new trend that all governments are directing research into simply because they are environment friendly and cheap. All researchers predict that the earth natural resources will run out and for the past 20 years have been trying to research new techniques to produce energy.\cite{SEIT2017} \\
\section{Problem Statement}
\label{sec:Problem Statement}
\indent Given data from small, private \& independent hydro power plants, we should be able to consolidate the data coming from their sensors and provide the owners with decision support such that the overall energy production is increased and down time is minimized. \\
\section{Goal and Approach}
\label{sec:GoalAndApproach}
\section{Original Contribution}
\label{sec:OriginalContribution}
\indent Currently there exists no system connecting small hydro power plants. Owners of said power plants don't communicate with each other resulting in a need for an early warning system (Decision support system). The main goal of the Decision support system is to receive data from previous power plants along the same river and direct the owner to do some action in response to the data received.\\
\section{Outline of the Thesis}
\label{sec:OutlineOfTheThesis}